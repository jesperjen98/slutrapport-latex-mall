%%% settings.tex --- 
%% 
%% Filename: settings.tex
%% Description: 
%% Author: Ola Leifler
%% Maintainer: 
%% Created: Tue Oct 19 21:11:31 2010 (CEST)
%% Version: $Id$
%% Version: 
%% Last-Updated: Tue Apr 25 08:49:48 2017 (+0200)
%%           By: Ola Leifler
%%     Update #: 43
%% URL: 
%% Keywords: 
%% Compatibility: 
%% 
%%%%%%%%%%%%%%%%%%%%%%%%%%%%%%%%%%%%%%%%%%%%%%%%%%%%%%%%%%%%%%%%%%%%%%
%% 
%%% Commentary: 
%% 
%% 
%% 
%%%%%%%%%%%%%%%%%%%%%%%%%%%%%%%%%%%%%%%%%%%%%%%%%%%%%%%%%%%%%%%%%%%%%%
%% 
%%% Change log:
%% 
%% 
%% RCS $Log$
%%%%%%%%%%%%%%%%%%%%%%%%%%%%%%%%%%%%%%%%%%%%%%%%%%%%%%%%%%%%%%%%%%%%%%
%% 
%%% Code:

%% To the author: Choose a citation style. Which one depends on your field of study.
%% Ask your supervisor or examiner if you are uncertain.

%% Good options for the style include for example:
%% APA: style=authoryear + \DeclareDelimFormat{nameyeardelim}{\addcomma\space}
%% Hardvard: style=authoryear
%% IEEE: style=ieee
%% Numeric: style=numeric,sorting=none  (like IEEE but with full names in the bibliography)
%% Most other styles
%%%
%% NOTE: You may need to add more changes to fully conform to a citation style
%% (making certain elements in italics, etc).

%% The default is very bad (numeric style sorted alphabetically instead of by appearance).

\usepackage[backend=biber,style=numeric,sorting=none,hyperref]{biblatex}
%% To set the font of your thesis, use the \setmainfont{} command,
%% surrounded with \ifxetex if you want to switch between xelatex and pdflatex
\ifxetex 
%\setmainfont [Scale=1]{Georgia}
\fi

%%%%%%%%%%%%
%% The VZ43 chapter style, from Memoir contributed chapter styles: ftp://ftp.tex.ac.uk/ctan%3A/info/MemoirChapStyles/MemoirChapStyles.pdf
%%%%%%%%%%%

\usepackage{calc,color}
\newif\ifNoChapNumber
\newcommand\Vlines{%
\def\VL{\rule[-2cm]{1pt}{5cm}\hspace{1mm}\relax}
\VL\VL\VL\VL\VL\VL\VL}
\makeatletter
\setlength\midchapskip{0pt}
\makechapterstyle{VZ43}{
\renewcommand\chapternamenum{}
\renewcommand\printchaptername{}
\renewcommand\printchapternum{}

\renewcommand\chapnumfont{\Huge\bfseries\centering}
\renewcommand\chaptitlefont{\Huge\bfseries\raggedright}
\renewcommand\printchaptertitle[1]{%
\Vlines\hspace*{-2em}%
\begin{tabular}{@{}p{1cm} p{\textwidth-3cm}}%
\ifNoChapNumber\relax\else%
\colorbox{black}{\color{white}%
\makebox[.8cm]{\chapnumfont\strut \thechapter}}
\fi
& \chaptitlefont ##1
\end{tabular}
\NoChapNumberfalse
}
\renewcommand\printchapternonum{\NoChapNumbertrue}
}
\makeatother

%% Customize the depth of the sections and subsections; too many numbered levels does not look good.
%% secnumdepth=2 or 3 work well, depending on your writing style

\setcounter{secnumdepth}{3} % \chapter=1, \section=2, \subsection=3 are numbered, but levels below this are not
\setcounter{tocdepth}{1} % \chapter=0, \section=1, \subsection=2 are included in the table of contents

%% To set bibliography options, refer to the biblatex manual and use
%% the ExecuteBibliographyOptions command below to set your options

\ExecuteBibliographyOptions{maxnames=99}
\ExecuteBibliographyOptions{maxcitenames=2} % Writes et. al for \citeauthor with more names than this

%% Change this to your appropriate BibTeX reference file (.bib)

\addbibresource{references.bib}

%%%%%%%%%%%%%%%%%%%%%%%%%%%%%%%%%%%%%%%%%%%%%%%%%%%%%%%%%%%%%%%%%%%%%%
%%% settings.tex ends here

%%% Local Variables: 
%%% mode: latex
%%% TeX-master: "demothesis"
%%% End: 
